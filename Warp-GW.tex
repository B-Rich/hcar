% Warp-GW.tex
\begin{hcarentry}{Warp}
\label{warp}
\report{Michael Snoyman}%05/13
%\status{stable}
\makeheader

Warp is a high performance, easy to deploy HTTP server backend for
WAI \cref{wai}.  Since the last HCAR, Warp has switched from enumerators to conduits~\cref{conduit}, added SSL support, and websockets integration.

Due to the combined use of ByteStrings, blaze-builder, conduit, and GHC's improved I/O manager, WAI+Warp has consistently proven to be Haskell's most performant web deployment option.

Warp is actively used to serve up most of the users of WAI (and Yesod).

``Warp: A Haskell Web Server'' by Michael Snoyman was published
in the May/June 2011 issue of IEEE Internet Computing:
\begin{compactitem}
\item
 Issue page: \url{http://www.computer.org/portal/web/csdl/abs/mags/ic/2011/03/mic201103toc.htm}
\item
 PDF: \url{http://steve.vinoski.net/pdf/IC-Warp_a_Haskell_Web_Server.pdf}
\end{compactitem}
\end{hcarentry}

\FurtherReading
  \url{https://github.com/snoyberg/posa-chapter/blob/master/warp.md}
\end{hcarentry}
