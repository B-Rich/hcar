% Persistent-GP.tex
\begin{hcarentry}{Persistent} 
\label{persistent}
\report{Michael Snoyman}%05/13
\participants{Greg Weber, Felipe Lessa}
\status{stable} 
\makeheader

Persistent is a type-safe data store interface for Haskell.
Haskell has many different database bindings available, but they provide few usefeul static guarantees.
Persistent uses knowledge of the data schema to provide a type-safe interface that re-uses existing database binding libraries. Persistent is designed to work across different databases, and works on Sqlite, PostgreSQL, MongoDB, and MySQL, with an experimental backend for CouchDB.

The 1.2 release features a refactoring of the module hierarchy. We're taking this opporunity to clean up a few idiosyncracies in the API and make the documentation a bit more helpful, but otherwise the library is remaining unchanged.

The MongoDB backend features new helpers, query operators, and bug fixes for working with embedded/nested models. One can store a list of Maps or records inside a column/field. This is required for proper usage of MongoDB. In SQL an embedded object is stored as JSON, which is convenient as long as the column is not queried.

In order to accomodate various different backend types, Persistent is broken up into multiple components (separated by type classes). There is one for storage/serialization, one for uniqueness, and one for querying. This means that anyone wanting to create database abstractions can re-use the battle-tested persistent storage/serialization layer without having to implement the full query interface.

Persistent's query layer is the same for any backend that implement the query interface, although backends can define their own additional operators. The interface is a straightforward usage of combinators:

\begin{code}
selectList [  PersonFirstName ==. "Simon", 
              PersonLastName ==. "Jones"] []
\end{code}

There are some drawbacks to the query layer: it doesn't cover every use case. Persistent has built-in some very good support for raw SQL. One can run arbitrary SQL queries and get back Haskell records or types for single columns. In addition, Felipe Lessa has created a library called esqueleto for having complete control over generating SQL but with type safety. persistent-MongoDB also has helpers for working with raw queries.

\FuturePlans 
Possible future directions for Persistent:
\begin{compactitem}
\item Adding key-value databases such as Redis without a query layer.
\item Full CouchDB support
\end{compactitem}

Persistent users may also be interested in Groundhog, a similar project.

Most of Persistent development occurs within the Yesod~\cref{yesod} community.
However, there is nothing specific to Yesod about it.
You can have a type-safe, productive way to store data,
even on a project that has nothing to do with web development.


\FurtherReading 
\url{http://www.yesodweb.com/book/persistent} 
\url{http://hackage.haskell.org/package/esqueleto}
\end{hcarentry}
